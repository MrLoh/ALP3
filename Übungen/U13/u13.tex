\documentclass[a4paper,11pt]{article}
\usepackage[ngerman]{babel}
\usepackage[ansinew]{inputenc}
\usepackage{amsmath}
\usepackage{geometry}
\usepackage{listings}

\title {ALP III: Datenstrukturen und Datenabstraktion\\ 13. Aufgabenblatt \\ �bungsgruppe 1.8: Marcel Erhardt } 
\author {Tobias Lohse/ Marvin Kleinert/ Anton Drewing}
\date{30.01.2015} 

\geometry{top=20mm, left=30mm, right=30mm, bottom=20mm}
\parindent 0pt


\lstset{numbers=left, numberstyle=\tiny, numbersep=5pt, tabsize=2}
\lstset{language=Python}


\begin{document}
\maketitle
\section*{Aufgabe 1}
\begin{enumerate}
\item[(a)] 3COLORING $\in$ NP\\
gegeben: ungerichteter Graph G, dargestellt durch Adjazenzliste\\
gesucht: F�rbung der Knoten von G, sodass jede Kante zwei unterschiedlich farbige Knoten verbindet\\
Zertifiakt c: Liste mit Farben der Knoten\\
Verifikator A:
\begin{lstlisting}
A(G,c):
	for edge in G:
		if c[edge.fst] == c[edge.snd]:
			return false;
	return true; 
\end{lstlisting}
Laufzeit von A: $O(n^2)$ mit $n=|V|$

\item[(b)] COMPOSITE $\in$ NP\\
gegeben: nat�rliche Zahl k in Bin�rdarstellung\\
gesucht: ganzzahliger Teiler von k\\
Zertifikat c: ganzzahliger Teiler von k\\
Verifikator A: 
\begin{lstlisting}
A(k,c):
	return (k mod c == 0);
\end{lstlisting}
Laufzeit: $O(n^2)$

\item[(c)] SAMECOMPONENTS $\in$ NP \\
gegeben: ungerichteter Graph $G = (V,E)$ und zwei Knoten $u,v\in V$\\
gesucht: Zusammenhangskomponente von G, in der u und v liegen\\
Zertifikat: Knotenliste von G, in der u und v liegen\\
Verifikator A: 
\begin{lstlisting}
A(<G,u,v>,c):
	uInC, vInC = false;
	for vertex in c:
		if u == vertex:
			uInC = true;
		if v == vertex:
			uInC = true;
	return (uInC && vInC);
\end{lstlisting}
Laufzeit: O(n) mit $n=|V|$

\end{enumerate}

\section*{Aufgabe 2}
Java-Code: siehe E-Mail-Abgabe

\section*{Aufgabe 3}
\begin{enumerate}
\item[(a)] SAMESUM $\in$ NP-vollst�ndig
\begin{enumerate}
\item[1.] SAMESUM $\in$ NP\\
gegeben: Folge $setA= a_1,...,a_n$ nat�rlicher Zahlen\\
gesucht: zwei Teilfolgen mit gleich gro�en Summen\\
Zertifikat c: zwei Teilfolgen $<c1,c2>$\\
Verifikator A:
\begin{lstlisting}
A(setA,<c1,c2>):
	sum1, sum2 = 0;
	for number in c1:
		sum1 += number;
	for number in c2:
		sum2 += number;
	return (sum1 == sum2);
\end{lstlisting}
Laufzeit: O(n)
\item[2.] SUBSET-SUM $\leq_p$ SAMESUM\\
Funktion f: $f(a_1,...,a_n,b)=(a_1,...,a_n,s+b,2s-b)$\\
$w \in SUBSET-SUM \Rightarrow f(w) \in SAMESUM$:\\
Wenn w in SUBSET-SUM liegt, gibt es eine Teilfolge von $a_1,...,a_n$, die summiert b ergibt. Diese bildet in f(w) mit 2s-b eine Menge, deren Summe 2s ist. Die �brigen Elemente der Folge ergeben summiert s-b, was zusammen mit dem Element s+b ebenfalls 2s ist.\\
$w \in SUBSET-SUM \Leftarrow f(w) \in SAMESUM$:\\
Wenn f(w) in SAMESUM liegt, muss eine Teilfolge von $a_1,...,a_n$ existieren , die b ergibt, da andernfalls eine Summe von Elementen aus f(w) gr��er und eine Summe kleiner als 2s ist.
\end{enumerate}

\item[(b)] VERTEX-COVER $\in$ NP-vollst�ndig
\begin{enumerate}
\item[1.] VERTEX-COVER $\in$ NP\\
gegeben: ungerichteter Graph $G=(V,E)$ und Zahl $k\in N$\\
gesucht: Menge $V'\subseteq V$, sodass jede Kante in E zu mind. einem Knoten in V' inzident ist\\
Zertifikat c: Menge V' mit Gr��e k\\
Verifikator A: 
\begin{lstlisting}
A(<(V,E),k>,c):
	if c.size != k:
		return false; 
	for edge in E:
		if !c.contains(edge.fst) && !c.contains(edge.snd):
			return false;
	return true;
\end{lstlisting}
Laufzeit: $(n-1)*n/2*n = O(n^3)$ mit $n=|V|$
\item[2.] CLIQUE $\leq_p$ VERTEX-COVER\\
Funktion f: $f(G,k)=(\overline{G},n-k)$\\
$ w \in CLIQUE \Rightarrow f(w) \in VERTEX-COVER $:\\
Wenn w in CLIQUE liegt, hat G eine Clique mit k Knoten. Im Komplement�rgraphen w�hlt man als V' die (n-k) Knoten, die nicht in der Clique sind, sodass alle Kanten in $\overline{G}$ mindestens zu einem Knoten aus V' inzident sind.\\
$ w \in CLIQUE \Leftarrow f(w) \in VERTEX-COVER $:\\
Wenn f(w) in VERTEX-COVER liegt, gibt es eine Menge V' mit (n-k) Knoten, sodass jede Kante in $\overline{E}$ zu mindestens einem Knoten in V' inzident ist. Zwischen den k Knoten, die nicht in V' liegen, darf es demnach keine Kanten untereinander geben, was im eigentlichen Graphen G dann einen vollst�ndigen Graphen mit k Knoten, also eine k-Clique zur Folge hat.
\end{enumerate}
\end{enumerate}
\end{document}