\documentclass[a4paper,11pt]{article}
\usepackage[ngerman]{babel}
\usepackage[ansinew]{inputenc}
\usepackage{amsmath}
\usepackage{geometry}

\title {ALP III: Datenstrukturen und Datenabstraktion\\ 2. Aufgabenblatt \\ �bungsgruppe 1.8: Marcel Erhardt } 
\author {Tobias Lohse/ Marvin Kleinert/ Anton Drewing}
\date{31.10.2014} 
\geometry{top=30mm, left=30mm, right=20mm, bottom=30mm}

\begin{document}
\maketitle

\section*{Aufgabe 1}
\subsection*{a)}
Algorithmus zum Bestimmen des gr��ten und zweitgr��ten Elementes:
\begin{itemize}
\item Vergleiche das erste und zweite Element; speichere den Wert des kleineren in der Variablen max2Elem und den Wert des gr��eren in maxElem
\item Durchlaufe die Folge ab dem dritten Element und vergleiche dabei jedes Element mit max2Elem und maxElem: 
\begin{itemize}
\item wenn das Element gr��er als max2Elem und kleiner als maxElem ist, speichere seinen Wert in max2Elem 
\item wenn das Element gr��er als max2Elem und maxElem ist, speichere den Wert von maxElem in max2Elem und den Wert des neuen Elementes in maxElem
\end{itemize}
\item Gib die Werte von max2Elem und maxElem zur�ck 
\end{itemize}

Anzahl der Vergleiche:
\[C(n)= \underbrace{1}_{elem_1 \leftrightarrow elem_2} +\underbrace{2*(n-2)}_{\forall elem_i, i\geq 3: elem_i \leftrightarrow max2Elem, elem_i \leftrightarrow maxElem} = 2n-3 \]

\subsection*{b)}
Idee des Algorithmus:
\begin{itemize}
\item Teile rekursiv Folgen der L�nge gr��er zwei in zwei H�lften
\item Vergleiche die Elemente der Zweierfolgen und sortiere diese aufsteigend
\item Vergleiche rekursiv jeweils die letzten zwei Elemente einer Folge mit denen der Nachbarfolge und sortiere die Elemente aufsteigend zu einer neuen Teilfolge
\item Gib nach der letzten Sortierung die letzten beiden Elemnete aus
\end{itemize}

Rekursionsgleichung:
\[C(1)=0\qquad C(2)=1\]
\[C(n)=2C(n/2)+4\]
\newline
L�sen der Rekursionsgleichung:
\begin{enumerate}
\item Einsetzen
\[ \begin{array}{ll}
C(n)&=2C(n/2)+4\\
&= 2*2*C(n/4)+8+4\\
&= 2*2*2*C(n/8)+16+8+4\\
&= \text{\dots}
\end{array} \]
\item Formel
\[ 2^k*C(n/2^k)+\sum_{i=1}^{k}{2*2^i} \]
\item k  f�r Anker = 2 bestimmen
\[ k = logn-1 \]
\item Anker und k einsetzen
\[ \begin{array}{ll}
C(n)&=2^{logn-1}*C(n/2^{logn-1})+\sum_{i=1}^{logn-1}{2*2^i} \\
& = 1/2n*1-2n-2+2*\sum_{i=0}^{logn}{2^i} \\
& = 1/2n-2n-2+2*\frac{1-2^{logn+1}}{1-2} \\
& = 1/2n-2n-2-2+2^{logn+2} \\
& = 5/2n-4
\end{array} \]
\end{enumerate}


Anzahl der Vergleiche:
\[C(n)=5/2n-4\]

\section*{Aufgabe 2}


\section*{Aufgabe 3}
\subsection*{a)}
Variante des Quicksort mit O(nlogn):
A sei deterministischer Algorithmus, der in O(n) den Median einer Eingabefolge S der L�nge n bestimmt.
\begin{itemize}
\item falls n == 1, gib n zur�ck, sonst
\item pivot = A(S);
\item durchlaufe Folge, bilde Teilfolgen $S_1$ aller Elemente $<$ pivot, und $S_2$ aller Elemente $>$ pivot
\item sortiere $S_1$ und $S_2$ rekursiv
\item gib aus $S_1$ sortiert, pivot, $S_2$ sortiert 
\end{itemize}

Anzahl der Vergleiche:
\[C(0)=0 \qquad C(1)=0\]
\[C(n)= C(\lceil n/2 \rceil -1) + C(\lfloor n/2\rfloor)+(n-1)+O(n)\]
Es gilt offensichtlich:
\[ C(\lceil n/2 \rceil-1) + C(\lfloor n/2 \rfloor)+(n-1)+O(n) \leq \underbrace{2*C(n/2)+n}_{nlog(n)} + O(n) \leq O(nlog(n)) \]

\subsection*{b)}
Deterministischer Linearzeit-Algorithmus f�r das Auswahlproblem:
\begin{itemize}
\item if n==1 then return Element von S
\item pivot = A(S)
\item spalte S auf in $S_<$, $S_=$, $S_>$
\item if $|S_<|\geq k$ then return quickSelect($S_<$,k)
\item if $k \leq |S_<|+|S_=|$ then return pivot;
\item quickSelect$(S_>,k-|S_<|-|S_=|)$ 
\end{itemize}

Laufzeit:
\[T(n)=(n-1)+\begin{cases} T( n/2) & \text{f�r } k\neq n/2 \\ 0 & \text{f�r } k=n/2 \end{cases} \]
Daraus folgt:
\[T(n) \leq n+T(n/2)+O(n) \leq 2n-1+O(n) \]

\end{document}